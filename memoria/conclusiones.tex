%!TEX root = tfg_fiesta.tex

El presente proyecto nos ha permitido implementar en Maude un entorno interactivo muy sencillo de utilizar que nos permite trabajar con la geometría Euclídea clásica con regla y compás. Pero no solo eso, sino que nos ha proporcionado una gran cantidad de módulos distintos con sus correspondientes test de unidad, el principal objetivo de este trabajo de fín de carrera. Por otro lado hemos podido desarrollar los módulos de aprendizaje, disponibles en el repositorio, los cuales facilitarán la tarea de aquellos que decidan iniciarse en este nuevo lenguaje mediante númerosos casos. \par

Sin embargo, y sin contradecir lo anterior, este proyecto no se encuentra cerrado, sino que se presta con mucha facilidad a seguir expandiéndolo. Este trabajo futuro dependerá del camino que se quiera seguir; las opciones más sencillas pasarían por la expansión de esta geometría a las tres dimensiones, pudiendo trazar en ella planos y esferas. Sin embargo esta no es la única posibilidad, pudiendo también dirigirse a un ambiente más cercano a la geometría diferencial mediante la creación de curvas y posteriormente superficies. Todas estas opciones nos permitirían continuar con la creación de Test de unidad. Sin embargo, se podría optar también por demostar la corrección y la completitud de los módulos y aprovechar las reglas dadas para crear un constructor automático de figuras con regla y compás.
