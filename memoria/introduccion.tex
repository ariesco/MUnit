%!TEX root = tfg_fiesta.tex

En esta memoria se presenta el lenguaje declarativo Maude a lo largo de diversos ejemplos y test de unidad. Valiendonos de estos últimos aprovecharemos también para crear distintos módulos de aprendizaje para aquellos que deseen iniciarse en este lenguaje.\par

A lo largo del Grado, más concretamente en la rama de Matemática Computacional, hemos estudiado diversos lenguajes de tipos muy diferentes, comenzando por Python o Pascal y llegando a Haskell y Prolog, que poco tienen que ver con los primeros. El porqué es lógico, los primeros estan orientados a objetos mientras los últimos son funcional y lógico respectivamente. Así pues, el aprendizaje de otro lenguaje es siempre una propuesta aceptada de buen grado, y Maude, del que yo no había oido hablar hasta entonces, parecía interesante.\par

Pero no todo es aprender un nuevo lenguaje, eso en sí mismo no es un proyecto, sin embargo la creación de los test de unidad si posee una consistencia suficiente. Estos son los otros desconocidos del proyecto que aquí se expone, pues aunque de forma abstracta hayamos podido trabajar con el ellos, en el sentido de probar manualmente las distintas funciones, la verdad es que el mayor contacto que se tiene con test de unidad propiamente dichos es en asignaturas en que nuestros programas deben pasarlos, pero nunca llegamos a implementarlos. \par

Ahora ya sí, con estos dos objetivos en mente, se debe crear un módulo, o en nuestro caso una serie de estos, que nos muestre claramente que es Maude, como funciona, y que al mismo tiempo nos permita crear una buena variedad de test. Aquí es importante decir que Maude es un lenguaje algebraico, y como tal, gran parte de los ejemplos que se pueden encontrar consisten en distintos modelos de este tipo, como anillos o cuerpos. Debido a esto se eligió algo distinto, que fuese claro al verse desde fuera, que puediese tener una aplicación futura, y que ademas nos permitiese crear test de unidad lo suficientemente variados y que no resultara demasiado abstracto para implementar un método de aprendizaje. Por estos motivos se decidió implementar la geometría clásica en dos dimensiones, y posteriormente un sistema de entrada salida que emulase las construcciones geométricas con regla y compás. \par

Así pues en los siguientes capítulos se procederá a explicar Maude y la creación de los módulos dedicados herramientos que crearemos así como los ya mencionados test de unidad y módulos de aprendizaje. En el ~\hyperref[cap.2]{capítulo 2} se explicará que es Maude con sus dos módulos principales de programación y numerosos ejemplos, a continuación presentaremos los test de unidad explicando las herramientas de que disponemos y de nuevo dando ejemplos. Concluyendo el capítulo explicaremos en que consiste el método de aprendizaje y su relación con otros ya existentes para otros lenguajes. Ya en el \hyperref[cap.3]{capítulo 3} comenzaremos a programar los modulos de geometría en dos dimensiones comenzando por puntas y rectas, y concluyendo estos en el \hyperref[cap.4]{capítulo 4} con la definición de las circunferencias y un último módulo llamado geo2d que se encargara de juntar todas las figuras anteriores. Despues de esto, en el \hyperref[cap.5]{capítulo 5} daremos algunos casos incompletos para el método de aprendizaje de estos módulos. Con la idea de la entrada y la salida, emulando la regla y el compás, creamos el \hyperref[cap.7]{capítulo 7}, pero antes creamos en el \hyperref[cap.6]{capítulo 6} los diccionarios de figuras, que serán la estructura de datos que utilizaremos en el séptimo para almacenar la información. Finalmente, en el \hyperref[cap.8]{capítulo 8}, se darán unas conclusiones finales y se expondrán las distintas maneras de continuar con el proyecto, ya sea extendiendo la geometría, rcreando un construcctor automático, o simplemente demostrando la corrección y la completitud de las funciones implementadas.
