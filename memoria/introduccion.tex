%!TEX root = tfg_fiesta.tex

En esta memoria se presenta el lenguaje declarativo Maude a lo largo de diversos ejemplos y sus correspondientes test de unidad. Valiéndonos de estos últimos aprovecharemos también para crear distintos módulos de aprendizaje para aquellos que deseen iniciarse en este lenguaje.\par

A lo largo del Grado en Ciencias Matemáticas, más concretamente en la rama de Matemática Computacional, hemos estudiado diversos lenguajes de tipos muy diferentes, comenzando por Python y Pascal y llegando a Haskell y Prolog, que poco tienen que ver con los primeros. Los primeros son lenguajes imperativos mientras los últimos son declarativos. Así pues, Maude, un lenguaje para mí desconocido resultaba una propuesta interesante.\par

Además para dotar de mayor consistencia al proyecto, no solo se implementará el lenguaje mediante implementación de ejemplos, también se desarrollarán test de unidad que permitirán evalura los conocimientos de futuros estudiantes. Estos son los otros desconocidos del proyecto que aquí se expone, pues aunque de forma abstracta hayamos podido trabajar con ellos, en el sentido de probar manualmente las distintas funciones, la verdad es que el mayor contacto que se tiene con test de unidad propiamente dichos es en asignaturas en que nuestros programas deben pasarlos, pero nunca llegamos a implementarlos. \par

Ahora sí, con estos dos objetivos en mente, se decidío crear una serie de módulos, que nos mostraran claramente qué es Maude, cómo funciona, y que al mismo tiempo nos permitiesen crear una buena variedad de pruebas. Aquí es importante decir que Maude es un lenguaje algebraico, y como tal, gran parte de los ejemplos que se pueden encontrar consisten en distintos modelos de este tipo, como anillos o cuerpos. Debido a esto se eligió algo distinto, que fuese claro al verse desde fuera, que puediese tener una aplicación futura, y que además nos permitiese crear test de unidad lo suficientemente variados y que no resultara demasiado abstracto para implementar un método de aprendizaje. Por estos motivos se decidió implementar la geometría clásica en dos dimensiones, y posteriormente un sistema de entrada/salida que emulase las construcciones geométricas con regla y compás. \par

Así pués, en los siguientes capítulos se procederá a explicar Maude y la creación de los módulos dedicados que crearemos así como los ya mencionados test de unidad y módulos de aprendizaje. En el capítulo~\ref{cap.2} se explicará qué es Maude con sus dos módulos principales de programación y numerosos ejemplos, a continuación presentaremos los test de unidad explicando las herramientas de que disponemos y de nuevo dando ejemplos. Concluyendo el capítulo explicaremos en que consiste el método de aprendizaje y su relación con otros ya existentes para otros lenguajes. En el capítulo~\ref{cap.3} comenzaremos a programar los módulos de geometría en dos dimensiones comenzando por puntos y rectas, y concluyendo estos en el capítulo~\ref{cap.4} con la definición de las circunferencias y un último módulo llamado \texttt{GEO2D} que se encargará de juntar todas las figuras anteriores. Después de esto, en el capítulo~\ref{cap.5} daremos algunos casos incompletos para el método de aprendizaje de estos módulos. Con la idea de la entrada y la salida, emulando la regla y el compás, creamos el capítulo~\ref{cap.7}, pero antes creamos en el capítulo~\ref{cap.6} los diccionarios de figuras, que serán la estructura de datos que utilizaremos en el séptimo para almacenar la información. Finalmente, en el capítulo~\ref{cap.8}, se darán unas conclusiones finales y se expondrán las distintas maneras de continuar con el proyecto, ya sea extendiendo la geometría, creando un constructor automático, o simplemente demostrando la corrección y la completitud de las funciones implementadas.\par
