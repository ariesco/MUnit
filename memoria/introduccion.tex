%!TEX root = tfg_fiesta.tex

En esta memoria se presenta el lenguaje funcional Maude a lo largo de diversos módulos que servirán como exposición de este. Por supuesto no nos encargaremos unicamente de esto, sino que iremos creando distintos test de unidad para las funciones que creemos, y aprovecharemos para crear distintos módulos de aprendizaje para aquellos que deseen iniciarse en este lenguaje.\par

A lo largo del Grado, más concretamente en la rama de Matemática Computacional, hemos estudiado diversos lenguajes de tipos muy diferentes, comenzando por Python o Pascal y llegando a Haskell y Prolog, que poco tienen que ver con los primeros. El porqué es lógico, los primeros estan orientados a objetos mientras los últimos son funcional y lógico respectivamente. Así pues, el aprendizaje de otro lenguaje es siempre una propuesta aceptada de buen grado, y Maude, del que yo no había oido hablar hasta entonces, parecía interesante.\par

Pero no todo es aprender un nuevo lenguaje, eso en sí mismo no es un proyecto, sin embargo la creación de los test de unidad si posee una consistencia suficiente. Estos son los otros desconocidos del proyecto que aquí se expone, pues aunque de forma abstracta hayamos podido trabajar con el ellos, en el sentido de probar manualmente las distintas funciones, la verdad es que el mayor contacto que se tiene con test de unidad propiamente dichos es en asignaturas en que nuestros programas deben pasarlos, pero nunca llegamos a implementarlos. \par

Ahora ya sí, con estos dos objetivos en mente se debe crear un módulo, o en nuestro caso una serie de estos, que nos muestre claramente que es Maude, como funciona, y que al mismo tiempo nos permita crear una buena variedad de test. Aquí es importante decir que Maude es un lenguaje algebraico, y como tal, gran parte de los ejemplos que se pueden encontrar consisten en distintos modelos de este tipo, como anillos o cuerpos. Debido a esto se eligió algo distinto, que fuese claro al verse desde fuera, que puediese tener una aplicación futura, y que ademas nos permitiese crear test de unidad lo suficientemente variados y que no resultara demasiado abstracto para implementar un metodo de aprendizaje. Por estos motivos se decidió implementar la geometría clásica en dos dimensiones, y posteriormente un sistema de entrada salida que emulase las construcciones geométricas con regla y compás. \par

Así pues a lo largo de las siguientes paginas se procederá a la creación de los módulos dedicados a las distintas figuras, puntos, rectas y circunferencias, así como los test que se encarguen de verificar todas sus funciones, y finalmente se dará un ultimo módulo, mucho más elavorado, encargado de esa emulación comentada anteriormente y que se aprovechara de los distintos módulos que Maude nos permite programar, funcionales y de sistema, los cuales se encontrarán explicados y con ejemplos en el capítulo siguiente.\par

De forma paralela se irán implementando los ya comentados módulos de aprendizaje, que haran uso de los normales y de los test para facilitar el acercamiento a Maude.\par







