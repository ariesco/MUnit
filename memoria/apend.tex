%!TEX root = tfg_fiesta.tex

\section{Punto de corte entre dos rectas}\label{r-r}

A continuación se mostrarán los cálculos referentes al punto de corte entre dos rectas dadas de la forma $y = mx + n$ y $y = m'x + n'$: \par

Igualamos una a la otra por la coordenada $y$ y despejamos $x$:\par

$$mx+n = m'x+n'$$
$$(m-m')x = n'-n$$
$$x = \frac{(n'-n)}{(m-m')}$$

Con la $x$ calculada sustituimos en la ecuación de una de las rectas y terminamos:\par

$$y = mx+n$$
$$y = \frac{m((n'-n)}{(m-m')) + n}$$

\section{Punto de corte entre una recta y una circunferencia}\label{r-c}

A continuación se mostrarán los cálculos referentes al punto de corte entre una recta y una circunferencia dadas de la forma: $y = mx + n$ y $(x-x')^2 + (y -y')^2 = r^2$ siendo $(x', y')$ el centro de la circunferencia y $r$ su radio: \par

Empezamos desarrollando los cuadrados en la ecuación de la circunferencia:\par

$$(x-x')^2 + (y -y')^2 = r^2$$
$$x^2 + x'^2 - 2xx' + y^2 + y'^2 - 2yy' = r^2$$
$$x^2 - 2xx' + y^2 + 2yy' = r^2 - x'^2 - y'^2$$

Sustituimos ahora la recta en la ecuación:\par


$$x^2 - 2xx' + (mx + n)^2 + 2(mx + n)y' = r^2 - x'^2 - y'^2$$
$$x^2 - 2xx' + mx^2 + n^2 + 2mnx + 2(mx + n)y' = r^2 - x'^2 - y'^2$$

Agrupamos las $x$ :\par


$$x^2 + mx^2 - 2xx' + 2mnx + 2mxy' + 2ny' + n^2 = r^2 - x'^2 - y'^2$$
$$(1+m)x^2 + x( 2mn + 2my' - 2x' ) + (2ny' + n^2 - r^2 + x'^2 + y'^2) = 0$$

Resolvemos la ecuación que nos queda y obtendríamos dos valores de $x$, uno de los cuales, denotado por $x''$ sustituye a $x$ en la ecuación de la recta:\par

$$y = mx'' + n$$

Lo que nos dará el valor de $y$ correspondiente para cada $x$.\par

\section{Punto de corte entre dos circunferencias}\label{c-c}

A continuación se mostrarán los cálculos referentes a los puntos de corte entre dos circunferencias dadas de la forma $(x-x')^2 + (y -y')^2 = (r')^2$ y $(x-x'')^2 + (y -y'')^2 = (r'')^2$: \par

En ambas desarrollamos los cuadrados:\par

$$x^2 + x'^2 - 2xx' + y^2 + y'^2 - 2yy' = (r')^2$$
$$x^2 + x''^2 - 2xx'' + y^2 + y''^2 - 2yy'' = (r'')^2$$

Le restamos la segunda a la primera:\par

$$x'^2 - 2xx' + y'^2 - 2yy' -(x''^2 - 2xx'' + y''^2 - 2yy'') = (r')^2 - (r'')^2$$

Agrupamos los términos según multipliquen a la $x$ o a la $y$:\par

$$ 2xx'' - 2xx' + 2yy'' - 2yy' - x''^2 + x'^2 - y''^2 + y'^2 + (r'')^2 - (r')^2 = 0$$
$$ 2x(x'' - x') + 2y(y'' - y') - x''^2 + x'^2 - y''^2 + y'^2 + (r'')^2 - (r')^2 = 0$$
$$ 2x(x'' - x') = 2y(y' - y'') + x''^2 - x'^2 + y''^2 - y'^2 - (r'')^2 + (r')^2 $$
$$ x =\frac{y(y' - y'')}{x'' - x'} + \frac{(x''^2 - x'^2 + y''^2 - y'^2 - (r'')^2 + (r')^2)}{2(x'' - x')} $$

Por comodidad al ser todo constantes haremos los siguientes cambios de variable:\par

$$B = \frac{y' - y''}{x'' - x'}$$
$$A = \frac{x''^2 - x'^2 + y''^2 - y'^2 - (r'')^2 + (r')^2}{2(x'' - x')}$$

Lo cual nos dejaría lo siguiente:\par

$$ x = yB + A$$

Sustituimos la $x$ en la ecuación de una de las circunferencias:\par

$$(yB + A-x')^2 + (y -y')^2 = (r')^2$$

Desarrollamos los cuadrados y agrupamos:\par

$$(yB + A)^2 + x'^2 - 2(yB + A)x' + y^2 + y'^2 - 2yy' = (r')^2$$
$$ (yB)^2 + A^2 + 2yAB + x'^2 - 2(yB + A)x' + y^2 + y'^2 - 2yy' = (r')^2$$
$$ (yB)^2 + y^2 + 2yAB - 2(yB + A)x' - 2yy' + A^2 + x'^2 + y'^2 - (r')^2 = 0$$
$$ (B^2 + 1)y^2 + 2y2AB - (B + A)x' - y') + (A^2 + x'^2 + y'^2 - (r')^2) = 0$$

Resolvemos la ecuación de segundo grado para obtener dos valores de $y$, después sustituimos en la ecuación de la $x$ y obtenemos los puntos que estábamos buscando.\par
