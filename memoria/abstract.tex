%!TEX root = tfg_fiesta.tex

A lo largo de este proyecto de fin de grado crearemos distintos módulos de geometría euclidea con regla y compás en Maude acompañandolos de sus correspondientes test de unidad. Para ello comenzaremos dando unas nociones básicas sobre Maude, el lenguaje declarativo que vamos a utilizar, explicando entonces sus distintos módelos de programación, veanse, funcionales o de sistema. A continuación presentaremos mUnit, módulo auxiliar encargado de realizar los test de unidad, y explicaremos el uso y funcionamiento de todas las herramientas que nos proporciona. El proyecto se encuentra entonces dividido en tres partes. La primera de ella se corresponde la creación de los módulos referentes a la geometría, los cuales nos permitiran ver los dos tipos de módulos que podemos implementar Maude, incluido uno interactivo que nos permitirá emular la geometria con regla y compás. La segunda parte se encuentra formada por los test de unidad definidos para cada módulo, y dentro de ellos para cada función, de los definidos en la primera parte.Finalmente la tercera parte constará de unos módulos de aprendizaje, los cuales se apoyarán en todo lo anteriormente creado para facilitar a los que esten interesados en el lenguaje a comenzar con él. \par
