%!TEX root = tfg_fiesta.tex

\section{Diccionarios}

Completado ya el primer módulo con sus respectivos casos de test así como los de aprendizaje, damos comienzo a los de entrada salida. Sin embargo antes siquiera de ponernos con ellos hemos de plantearnos como funcionará a nivel interno. Nuestro objetivo es que todas las instrucciones esten ahí, así pues necesitaremos almacenar toda la información en algun sitio, para lo que necesitaremos algun tipo de estructura de datos. De forma paralela tendremos que ver el trato que le damos a las figuras, podemos tenerlas simplemente almacenadas tal cual son y llamarlas utilizando su definición. Sin embargo esto no es para nada cómodo y siguiendo el orden natural de las cosas consideraremos que las figuras tienen nombre, ya decidiremos cuando llegue el momento cual y como se le dá. \par

Continuando con la definición de las estructuras de datos, podriamos simplemente implementar una lista de pares nombre figura, pero para esto es mucho más util crear un diccionario. Los elementos del diccionario serán las entradas, \texttt{entry}, compuestas por un \texttt{Qid}, un identificador, que es un tipo ya definido en Maude, y la figura correspondiente, del tipo \{Figure}.\

Comencemos con la creación del diccionario: \par
{\codesize
\begin{verbatim}
fmod DICC-FIGURES is
   pr GEO2D .
   pr QID .

   sorts Dic Entry .
   subsorts Entry < Dic .

   op _->_ : Qid Figure -> Entry [ctor] .
   op mtD : -> Dic [ctor] .
   op _._ : Dic Dic -> Dic [ctor assoc comm id: mtD] .

\end{verbatim}
}

Como se puede ver la definición es similar a la de las listas, \texttt{mtD}es el diccionario vacio y \verb"_._" la operación que nos permite unir diccionarios, siendo \texttt{mtD} su elemento neutro. Sin embargo la propiedad más importante de estos constructores sera la commutatividad, que no facilitara enormemente las operaciones siguientes. Pensando en las que debemos implementar es natural que la primera sea aquella que dada un \texttt{Dic} y un \texttt{Qid} nos devuelva la fígura correspondiente. De forma paralela implementaremos tambien \verb"contains?" que nos indica si hay alguna entrada en el diccionario que corresponda con el \texttt{Qid} dado. \par

{\codesize
\begin{verbatim}
   var  D : Dic .
   vars Q Q' : Qid .
   vars F F' : Figure .

   op _[_] : Dic Qid ~> Figure .           op contains? : Dic Qid -> Bool .
   *** Dado un Qid devuelve la figura      *** Dado un Qid indica si el diccionario
   *** correspondiente.                    *** tiene una entreda que le corresponda.
   eq (D . Q -> F)[Q] = F .                eq contains?((Q -> F) . D, Q) = true .
                                           eq contains?(D, Q) = false [owise] .

\end{verbatim}
}

Continuando con las operaciones básicas, en determinados momentos puede ser necesario actualizar el diccionario, asi que para ello creamos una nueva funcion que ademas se encarga de evitar las repeticiones y que entradas con distintos \texttt{Qid} tengan asociada la misma figura. \par

{\codesize
\begin{verbatim}
   op _[_/_] : Dic Qid Figure -> Dic .
   eq (D . (Q' -> F) . (Q -> F'))[Q' / F'] = D . Q' -> F' . 
   ceq (D . (Q' -> r1) . (Q -> r2))[Q' / r2] = D . Q' -> r2
     if are-equal(r1, r2) .
   eq (D . Q -> F)[Q / F'] = D . Q -> F' .
   eq (D . Q -> F)[Q' / F] = D . Q -> F .
   ceq (D . Q -> r1)[Q' / r2] = D . Q -> r1
     if are-equal(r1, r2) .   
   eq D[Q / F] = D . Q -> F [owise] .
\end{verbatim}
}

Como se puede ver el único caso un poco distinto se corresponde con las rectas, ya que estas pueden ser distintas para Maude pero equivalentes para nosotros, así que hacemos uso de la función \verb"are-equal". Para terminar crearemos unas dos últimas funciones. La primera de ellas indica si una figura esta en un diccionario, y la segunda devuelve su nombre en caso afirmativo. Para el caso en que la figura no se encuentre en el diccionario tenemos varias opcciones, no devolver nada, extender los \texttt{Qid} con un nuevo elemento que signifique vacío, o utilizar el módulo \texttt{QidList} y de él \texttt{nil}, que es el constructor referido a lista de \texttt{Qid} vacia. Como no definir el caso nos puede dar mucho problemas al dejar la función parcialmente definida, se valoran los otros dos casos. Extender el módulo de \texttt{Qid} no es dificil, pero como existe la posibilidad de utilizar \texttt{nil} simplemente importando el módulo correspondiente haremos uso de esa opción. \par

{\codesize
\begin{verbatim}
   op _in_ : Figure Dic -> Bool .             op name : Dic Figure -> QidList .
   *** Indica si la figura pertenece al       *** Dado un diccionnario y una figura 
   *** diccionario.                           *** devuelve el identificador de esta.
   eq F in D . (Q -> F) = true .              eq name(D . Q -> F, F) = Q .
   eq F in D = false [owise] .                eq name(D , F) = nil [owise] . 
\end{verbatim}
}

\section{Diccionario de figuras}

Y con esto ponemos fín a los diccionarios para figuras. Estos, aunque funcionales a la hora de trabajar con ellos, carecen totalmente de órden, ya que los puntos, rectas y circunferencias se encuentran totalmente mezclados, lo cual teniendo tres clases bien diferenciadas resulta extraño. Así pues construiremos un nuevo tipo \texttt{DicFigures} que se corresponde con un conjunto de tres diccionarios, estando cada uno de ellos destinado a un tipo de figura comcreta. Para comenzar extendemos los tipos del módulo y le dotamos de un constructor para este nuevo tipo: \par

{\codesize
\begin{verbatim}
   sorts Dic Entry DicFigures .
   subsorts Entry < Dic .

   op [_ , _ , _] : Dic Dic Dic -> DicFigures [ctor] .

   vars  D Dp Dr Dc Dp' Dr' Dc' : Dic .
   vars  Q Q' : Qid .
   vars F F' : Figure .
   var p1 : Point .
   vars r1 r2 : Line .
   var c1 : Circumference .
\end{verbatim}
}

Las funciones que definiremos para este nuevo tipo serán exactamente las mismas que para los diccionarios normales, y utilizarán además los mismos comandos para mayor comodidad. \par

{\codesize
\begin{verbatim}
   op _._ : DicFigures Entry -> DicFigures .
   eq [Dp , Dr , Dc] . Q -> p1 = [(Dp . Q -> p1) , Dr , Dc] .
   eq [Dp , Dr , Dc] . Q -> r1 = [Dp , (Dr . Q -> r1) , Dc] .
   eq [Dp , Dr , Dc] . Q -> c1 = [Dp , Dr , (Dc . Q -> c1)] .

   op _[_] : DicFigures Qid ~> Figure .
   eq [Dp . Q -> p1 , Dr , Dc][Q] = p1 .
   eq [Dp , Dr . Q -> r1 , Dc][Q] = r1 .
   eq [Dp , Dr , Dc . Q -> c1][Q] = c1 . 

   op _[_/_] : DicFigures Qid Figure -> DicFigures .
   eq [Dp , Dr , Dc][Q / p1] = [Dp[Q / p1], Dr, Dc] .
   eq [Dp , Dr , Dc][Q / r1] = [Dp, Dr[Q / r1], Dc] .
   eq [Dp , Dr , Dc][Q / c1] = [Dp, Dr, Dc[Q / c1]] .

   op _in_ : Figure DicFigures -> Bool .
   eq F in [Dp . Q -> F, Dr , Dc] = true .
   eq F in [Dp , Dr . Q -> F, Dc] = true .
   eq F in [Dp, Dr , Dc . Q -> F] = true .
   eq F in [Dp, Dr, Dc] = false [owise] .

   op contains? : DicFigures Qid -> Bool .
   eq contains?([(Q -> F) . Dp, Dr, Dc], Q) = true .
   eq contains?([Dp, (Q -> F) . Dr, Dc], Q) = true .
   eq contains?([Dp, Dr, (Q -> F) . Dc], Q) = true .
   eq contains?([Dp, Dr, Dc], Q) = false [owise] .

   op name : DicFigures Figures -> QidList .
   eq name([Dp, Dr, Dc], p1) = name(Dp, p1) .
   eq name([Dp, Dr, Dc], r1) = name(Dr, r1) .
   eq name([Dp, Dr, Dc], c1) = name(Dc, c1) .
\end{verbatim}
}

Unas vez finalizado el módulo tendremos que implementar los casos de test correspondientes, y los módulos de aprendizaje. Mostramos entonces a continuación parte de estos dos: \par

{\codesize
\begin{verbatim}
(munit DICC-FIGURES is
      ***Veamos el operador _[_]
   assertEqual('p1 -> p(0.0 0.0) ['p1], p(0.0 0.0))
******* Falta un caso en el test, en concreto el siguiente
   assertDifferent('p1 -> p(0.0 0.0) ['p2], p(0.0 0.0))
      ***Veamos el operador _[_/_]
   assertEqual('p1 -> p(0.0 0.0) ['p1 / p(1.0 1.0)], 'p1 -> p(1.0 1.0))
   assertEqual('p1 -> p(0.0 0.0) ['p2 / p(1.0 1.0)], 
              ('p1 -> p(0.0 0.0)) . ('p2 -> p(1.0 1.0)))
   assertEqual('r1 -> r(p(0.0 0.0) p(1.0 1.0)) ['r2 / r(p(1.0 1.0) p(0.0 0.0))], 
              ('r1 -> r(p(0.0 0.0) p(1.0 1.0))))
      ***Veamos la funcion _in_
   assertTrue(p(0.0 0.0) in ('p1 -> p(0.0 0.0)) . ('p2 -> p(1.0 1.0)))
   assertTrue(p(0.0 0.0) in ('p1 -> p(0.0 0.0)) . ('p2 -> p(1.0 1.0)))
   assertFalse(p(0.0 1.0) in ('p1 -> p(0.0 0.0)) . ('p2 -> p(1.0 1.0)))
      ***Veamos la funcion contains?
   assertTrue(contains?(('p1 -> p(0.0 0.0)) . ('p2 -> p(1.0 1.0)), 'p1))
   assertTrue(contains?(('p1 -> p(0.0 0.0)) . ('p2 -> p(1.0 1.0)), 'p2))
   assertFalse(contains?(('p1 -> p(0.0 0.0)) . ('p2 -> p(1.0 1.0)), 'r1))
      ***Veamos la funcion name
   assertEqual(name('p -> p(0.0 0.0), p(0.0 0.0)), 'p)
   assertDifferent(name('p -> p(0.0 0.0), p(0.0 0.0)), 'r)
\end{verbatim}
}

{\codesize
\begin{verbatim}

Aquí irían los casos con huecos.

\end{verbatim}
}

Una vez concluido lo que consideramos necesario para el siguiente módulo es necesario aclarar que este no se encuentra terminado, nada más lejos, sí no que como ya se comentó en su momento, será utilizado como auxiliar para la entrada salida.
