%!TEX root = tfg_fiesta.tex

A lo largo de este y el proximo capítulo desarrollaremos tres módulos dedicados enteramente a la geometría en dos dimensiones. En este nos encargaremos de crear los puntos y las rectas con todas las funciones que se considerarán necesarias, así como de crear los correspondientes de unidad. Los módulos de aprendizaje por el contrario no aparecerán aquí, sino que podrán encontrarse en el repositorio online.


\section{POINT}

Comencemos con la creación del primer módulo, la definición de los puntos es sencilla, construyéndose como un par formado por dos números reales, igual que se haría en papel. Las funciones que hemos considerado necesarias por el momento son unicamente dos, la que nos calcula la distancia entre dos de ellos, y la que nos intercambia las coordenadas. Así pues el módulo funcional, quedaría como sigue.
{\codesize
\begin{verbatim}
fmod POINT is

   pr FLOAT .
   sorts Point .

   op p(__) : Float Float -> Point [ctor] .
   *** Define un punto como un par de numeros reales

   vars z11 z12 z21 z22 : Float .

   op distance : Point Point -> Float .
   *** Calcula la distancia entre dos puntos mediante la metrica usual.
   eq distance(p(z11 z12) , p(z21 z22)) = sqrt(((z21 - z11) ^ 2.0) + ((z22 - z12) ^ 2.0)) .

   op trans-point : Point -> Point .
   *** Intercambia las coordenadas de un punto
   eq trans-point(p(z11 z12)) = p(z12 z11) .

endfm
\end{verbatim}
}

El módulo queda entonces completo aunque resulta muy simple, sin embargo, como será necesario importarlo para la creación de los otros dos, lo utilizaremos también para las funciones auxiliares que vayamos necesitando en los módulos siguientes, como por ejemplo la implementación de las ecuaciones de segundo grado. A continuación daremos algunos casos del test de unidad:\par

{\codesize
\begin{verbatim}
(munit POINT is
assertEqual(distance(p(1.0 0.0), p(0.0 1.0)), sqrt(2.0))
assertEqual(distance(p(0.0 0.0), p(4.0 3.0)), sqrt(13.0))
assertEqual(distance(p(1.0 0.0), p(2.0 0.0)), 1.0)
assertEqual(distance(p(1.0 1.0), p(1.0 3.0)), 2.0)
assertDifferent(trans-point(p(1.0 0.0)), p(1.0 0.0))
assertEqual(trans-point(p(1.0 0.0)), p(0.0 1.0))
endu)
\end{verbatim}
}

\section{LINE}

Dando por finalizada la creación del \texttt{fmod POINT}, comenzaremos con el de las rectas. Estas se implementarán de forma análoga a los puntos, pero como un par de estos en vez de números reales. Como estamos siguiendo un camino similar al de la geometría básica, implementaremos además el tipo vector, que se utilizará en varias operaciones de las rectas. Estos se construirán de nuevo como un par de reales, que será el correspondiente a extremo menos origen de los puntos que las definen. Empezamos dando los constructores correspondientes:

{\codesize
\begin{verbatim}
fmod LINE is 

   pr POINT .
   sorts Line Vector .

   op r(__) : Point Point -> Line [ctor] .
   *** Define una recta como un par de puntos, mas adelante la daremos ecuaciones.
   op v(__) : Float Float -> Vector [ctor] .
   *** Define un vector como un par de reales.
   *** Extremo menos origen.

   vars p1 p2 p3 p4 : Point .
   vars z11 z12 z21 z22 z31 z32 z41 z42 m n : Float .
   vars r1 r2 : Line .
   vars v1 v2 : Vector .
   vars t : Bool .
\end{verbatim}
}

Las primeras funciones que crearemos para los vectores servirán para, o bien obtener información de ellos o construir otros. Estas se corresponden con \texttt{v-scalar-prod}, que nos devuelve el producto escalar de dos vectores, \texttt{v-norm}, que nos da la norma de un vector, \texttt{perpendicular-vector}, que nos proporciona un vector perpendicular a uno dado, y finalmente \texttt{v-are-perpendicular?}, que nos dice si dos vectores son perpendiculares. Todas estas funciones son sencillas, y practicamente una copia de sus definiciones, motivo por el que no se incluyen aquí. Lo que sí se mostrará es parte de su test de unidad, logicamente ejecutado sobre el módulo completo:\par

{\codesize
\begin{verbatim}
(munit LINE is
      ***Función v-scalar-prod.
   assertEqual(v-scalar-prod(v(1.0 0.0), v(0.0 1.0)), 0.0)
   assertEqual(v-scalar-prod(v(1.0 2.0), v(2.0 1.0)), 4.0)	
      ***Funcion v-norm.
   assertDifferent(v-norm(v(2.0 0.0)), 4.0)
   assertEqual(v-norm(v(2.0 3.0)), sqrt(13.0))
      ***Funcion perpendicular-vector.
   assertDifferent(perpendicular-vector(v(1.0 2.0)), v(2.0 1.0))
   assertEqual(perpendicular-vector(v(1.0 0.0)), v(0.0 -1.0))
      ***Funcion v-are-perpendicular?.
   assertTrue(v-are-perpendicular?(v(1.0 0.0), v(0.0 1.0)))
   assertFalse(v-are-perpendicular?(v(1.0 2.0), v(2.0 1.0)))
   assertTrue(v-are-perpendicular?(v(2.0 3.0), perpendicular-vector(v(2.0 3.0))))
endu)
\end{verbatim}
}

Antes de continuar con las operaciones para rectas construiremos dos funciones opuestas, \texttt{direction-vector-line} que nos devuelve el vector director de una recta, y \texttt{line-from-vector}, que recibe un vector, un punto y devuelve una recta: \par
{\codesize
\begin{verbatim}
   op direction-vector-line : Line -> Vector .
   *** Devuelve el vector director de una recta
   eq direction-vector-line(r(p(z11 z12) p(z21 z22))) =
      v((z21 - z11) (z22 - z12)) .

   op line-from-vector : Vector Point -> Line .
   *** Construye una recta mediante un vector y un punto dados.
   eq line-from-vector(v(z11 z12) , p(z21 z22)) =
      r(p(z21 z22) p((z11 + z21) (z12 + z22))) .
\end{verbatim}
}

Estas funciones nos ayudarán a comprobar si dos rectas son perpendiculares o a calcular una perpendicular a otra en función a un punto dado, ya que simplemente convertiremos las rectas a vectores, y con las funciones anteriores se realizará todo el proceso para después simplemente volver a convertirlos en rectas. Estas funciones son \texttt{perpendicular-line} y \texttt{are-perpendiculars} respectivamente. Volviendo a la idea original del módulo total, conseguir la geometría suficiente para imitar las construcciones con regla y compás, lo que necesitamos es hallar el punto de corte entre dos rectas. Para esto pondremos las ecuaciones en su forma implícita y resolveremos el sistema correspondiente. Comencemos entonces por encontrar los valores m y n, pendiente y desplazamiento respectivamente, de las rectas: \par

{\codesize
\begin{verbatim}
   op equ-line-m : Line -> Float .
   *** Dados los dos puntos que definen la recta, devuelve el valor de m, 
   *** para poder "construirla" de forma y = mx + n.
   eq equ-line-m(r(p(z11 z12) p(z21 z22))) = (z22 - z12)/(z21 - z11) .

   op equ-line-n : Line -> Float .
   *** Dados los dos puntos que definen la recta, devuelve el valor de n, 
   *** para poder "construirla" de forma y = mx + n , teniendo cuidado 
   *** en las que son de la forma  x = n
   eq equ-line-n(r(p(z11 z12) p(z21 z22)))= z12 - 
      (z11 * equ-line-m(r(p(z11 z12) p(z21 z22)))) .

   op are-equal : Line Line -> Bool .
   *** comprueba si dos rectas son iguales a partir de m y n, 
   *** pendiente y desplazamiento, pues una recta
   *** puede construirse mediante infinitos pares de puntos.
   eq are-equal(r(p(z11 z12) p(z11 z22)), r(p(z11 z32) p(z11 z42))) = true . 
   eq are-equal(r1 , r2) = (equ-line-m(r1) == equ-line-m(r2)) and 
      (equ-line-n(r1) == equ-line-n(r2)) [owise] .
\end{verbatim}
}

Como se puede ver, además de las funciones antes mencionadas se ha incluido una más llamada \texttt{are-equal}, ya que la igualdad básica, \verb"_==_", en estos casos falla pues la misma recta puede estar definida de infinitas maneras distintas, véase: \texttt{r(p(0.0 0.0) p(1.0 1.0))} y \texttt{r(p(0.0 0.0) p(n n))} son la misma recta para todo \texttt{n} natural, pero no son identicamente iguales, que es lo que Maude necesita, así que para él son distintas. Antes de definir los puntos de corte de dos rectas necesitaremos dos funciones básicas para la distinción de casos, \texttt{horizontal} y \texttt{vertical} que nos indicarán respectivamente si la recta que les demos es horizontal o vertical. Como ya sucedió más arriba la definición de estas funciones es practicamente trivial, simplemente comprobamos las coordenadas de los puntos, así que no se incluirán aquí.\par
Aprovechando la creación de estas funciones realizaremos la última función menor, \texttt{are-in-line} que indica si un punto pertenece a una recta comprobando la distancia del punto dado con los dos que la definen, si la suma de las distancias del punto dado a los originales es igual que la distancia entre estos el punto estará contenido en la recta. Esta función tampoco aparecerá aquí por los motivos antes mencionados, su ecuación es practicamente trivial, consistiendo en una transcripción de lo que acabo de contar.\par

A continuación se muestran algunos casos del test de unidad correspondiente a las funciones para rectas. \par

{\codesize
\begin{verbatim}
(munit LINE is
   assertEqual(direction-vector-line(r(p(0.0 0.0) p(1.0 1.0))), v(1.0 1.0))	
   assertDifferent(line-from-vector(v(1.0 2.0), p(0.0 0.0)),r(p(1.0 2.0) p(0.0 0.0)))
   assertTrue(are-equal(line-from-vector(v(1.0 2.0), p(0.0 0.0)), r(p(0.0 0.0) p(1.0 2.0))))
   assertTrue(are-equal(perpendicular-line(r(p(0.0 0.0) p(0.0 1.0)), p(0.0 0.0)), r(p(0.0 0.0) p(1.0 0.0))))
   assertTrue(are-perpendiculars(r(p(0.0 0.0) p(0.0 1.0)), r(p(0.0 0.0) p(1.0 0.0))))
   assertFalse(are-perpendiculars(r(p(0.0 0.0) p(1.0 1.0)), r(p(1.0 1.0) p(1.0 2.0))))
   assertEqual(equ-line-m(r(p(0.0 0.0) p(1.0 1.0))), 1.0)
   assertEqual(equ-line-n(r(p(1.0 1.0) p(2.0 0.0))), 2.0)
   assertTrue(are-equal(r(p(0.0 0.0) p(1.0 1.0)) , r(p(23.0 23.0) p(-1.0 -1.0))))
   assertTrue(vertical(r(p(13.0 -43.0) p(13.0 56.0))))
   assertTrue(horizontal(r(p(57.0 12.0) p(24.0 12.0))))
   assertTrue(are-in-line(p(0.0 0.0), r(p(0.0 0.0) p(1.0 1.0)))))
endu)
\end{verbatim}
}

Llegamos ahora a la parte más interesante de este módulo, el punto de corte de dos rectas.
Por comodidad lo primero que se hará será crear una función que nos dirá si dos rectas se cortan y
posteriormente crearemos otra que nos devuelva ese punto. En caso de que las rectas no se corten, la función que nos da el punto de corte nos devolverá un error de algún tipo, pero eso será algo que se arreglará más adelante. \par

{\codesize
\begin{verbatim}	
   op cut?-r-r : Line Line -> Bool .
   *** comprueba si dos rectas se cortan a partir de la pendiente de estas
   *** devolviendo el valor false si son paralelas y true en caso contrario.
   *** El caso en que sean coincidentes debera verse de forma manual comparandolas.
   ceq cut?-r-r(r1, r2) = true 
   if vertical(r1) and horizontal(r2) .
   ceq cut?-r-r(r1, r2) = true 
   if vertical(r2) and horizontal(r1) .
   eq cut?-r-r(r1, r2) = not (equ-line-m(r1) == equ-line-m(r2)) .
\end{verbatim}
}

Para hallar el punto de corte haremos lo mencionado anteriormente, con las dos rectas en implícitas devolvemos la solución del sistema correspondiente. El método es trivial, salvo los cálculos, pero al encontrarnos en un caso general debemos hacer una distinción de casos que se corresponden con las rectas horizontales y verticales, ya que la pendiente cero e infinito da bastantes problemas. \par

{\codesize
\begin{verbatim}	
   op cut-point-r-r : Line Line -> Point .
   *** Devuelve el punto de corte de dos rectas dadas.
   ceq cut-point-r-r(r1, r2) = cut-point-r-r(r2, r1)
   if vertical(r1) .
   ceq cut-point-r-r(r1, r2) = cut-point-r-r(r2, r1)
   if horizontal(r2) .
   ceq cut-point-r-r(r1, r2) = p(x y)
   if x := cut-point-x(r1, r2) /\
      y := cut-point-y(r1, r2) .

   op cut-point-x : Line Line -> Float .
   *** Devuelve la coordenada X del punto de corte de las dos rectas.
   ceq cut-point-x(r1, r2) = (n2 - n1) / (m1 - m2)
   if not vertical(r2) /\
      m1 := equ-line-m(r1) /\
      m2 := equ-line-m(r2) /\
      n1 := equ-line-n(r1) /\
      n2 := equ-line-n(r2) .
   eq cut-point-x(r1, r(p(z11 z12) p(z11 z22))) = z11 .
	
   op cut-point-y : Line Line -> Float .
   *** Devuelve la coordenada Y del punto de corte de las dos rectas.
   ceq cut-point-y(r1, r2) = m1 * x + n1
   if not horizontal(r1) /\
      m1 := equ-line-m(r1) /\
      n1 := equ-line-n(r1) /\
      x := cut-point-x(r1, r2) .
   eq cut-point-y(r(p(z11 z12) p(z21 z12)), r2) = z12 .
\end{verbatim}
}

Los cálculos de las funciones \texttt{cut-point-x} y \texttt{cut-point-y} se pueden encontrar en el apéndice correspondiente junto con el de otras funciones que veremos más adelante. Al igual que hemos hecho en el resto de casos comprobamos que todo funciona correctamente antes de crear el módulo incompleto. Sin embargo, a la hora de ejecutar los test nos encontramos con un único caso que no pasa, lo cual resulta extraño, que es el siguiente: \par

{\codesize
\begin{verbatim}
(munit LINE is
   assertEqual(cut-point-r-r(r(p(2.0 3.0) p(7.0 5.0)), r(p(9.3 2.4) p(7.0 5.0))),
              p(7.0 5.0))
end)
\end{verbatim}
}

Cuya ejecución, que como se puede ver está forzada para dar \texttt{true}, es la siguiente:\par

{\codesize
\begin{verbatim}
1 test cases were executed.
1 failures.

assertEqual(cut-point-r-r(r(p(2.0 3.0)p(7.0 5.0)),r(p(9.3000000000000007
    2.3999999999999999)p(7.0 5.0))),p(7.0 5.0)) failed.
\end{verbatim}
}

Para ver el problema calcularemos el punto por separado, pues, por como están construidas las rectas, debería devolvernos el punto \texttt{p(7.0 5.0)} .\par
{\codesize
\begin{verbatim}
Maude> red cut-point-r-r(r(p(2.0 3.0) p(7.0 5.0)), r(p(9.3 2.4) p(7.0 5.0))) .
reduce in GEO2D : cut-point-r-r(r(p(2.0 3.0) p(7.0 5.0)), r(p(
    9.3000000000000007 2.3999999999999999) p(7.0 5.0))) .
rewrites: 88 in 0ms cpu (0ms real) (~ rewrites/second)
result Point: p(7.0000000000000009 5.0000000000000009)
\end{verbatim}
}
Aquí vemos el error, Maude no se comporta bien con los decimales, y aunque sea en una posición tan baja, la dieciséis, esos números no son ni siete ni cinco, así que Maude nos devuelve un fallo. Para corregir este tipo de errores incluiremos en el \verb"fmod POINT" la primera función auxiliar, que dado dos números dirá si son iguales en base a un Epsilon que podremos elegir.\par
{\codesize
\begin{verbatim}
   op equal-epsilon : Float Float Float -> Bool .
   *** Comprueba si dos numeros reales son iguales en funcion de un epsilon.
   ceq equal-epsilon(z11, z12, a) = equal-epsilon(z12, z11, a)
   if z11 < z12 .
   ceq equal-epsilon(z11, z12, a) = true
   if (z11 >= z12) 
      /\ a > (z11 - z12) .
   ceq equal-epsilon(z11, z12, a) = false
   if (z11 >= z12) 
      /\ a <= (z11 - z12) .
\end{verbatim}   
}

A continuación veremos, haciendo un pequeño cambio, como el test que antes fallaba ahora nos devuelve un resultado correcto.
{\codesize
\begin{verbatim}
(munit LINE is
   assertEqual(cut-point-r-r(r(p(2.0 3.0) p(7.0 5.0)), r(p(9.3 2.4) p(7.0 5.0))),
               p(7.0 5.0))
   assertTrue(equal-epsilon(0.0, distance(cut-point-r-r(r(p(2.0 3.0) p(7.0 5.0)),
              r(p(9.3 2.4) p(7.0 5.0))), p(7.0 5.0)), 10.0 ^ (-5.0)))
endu)

2 test cases were executed.
1 failures.

assertEqual(cut-point-r-r(r(p(2.0 3.0)p(7.0 5.0)),r(p(9.3000000000000007
    2.3999999999999999)p(7.0 5.0))),p(7.0 5.0)) failed.
assertTrue(equal-epsilon(0.0,distance(cut-point-r-r(r(p(2.0 3.0)p(7.0 5.0)),r(
    p(9.3000000000000007 2.3999999999999999)p(7.0 5.0))),p(7.0 5.0)),1.0e+1 ^
    -5.0)) passed.
\end{verbatim} 
}
Se puede ver que todo ha pasado a funcionar correctamente, aunque a veces los decimales nos pueden dar problemas. De esta manera ponemos fin al módulo\texttt{fmod LINE}. \par
